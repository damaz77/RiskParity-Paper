\begin{proposition}
Suppose that the level set $\mathcal{L}_0$ is a compact set. Let $\{(x^k, y^k)\}$ be the sequence of points generated by the decomposition algorithm. Then
\begin{itemize}
\item $(x^k, y^k) \in \mathcal{L}_0 \enskip \forall k$ 
\item  $\{(x^k, y^k)\}$ admits limit points and each limit point is a solution for Problem (\ref{eq:problem})
\end{itemize}
\end{proposition}
\begin{proof}
From (\ref{eq:updatey}) we have
\begin{equation}
f(x^{k}, y^{k+1}) \leq f(x^{k}, y^{k})
\end{equation}
and from the update rule of $x$ it follows (thanks to the Armijo properties)
\begin{equation}\label{eq:dec}
f(x^{k+1}, y^{k+1}) \leq f(x^{k}, y^{k+1})
\end{equation}
Finally we have
\begin{equation}
f(x^{k+1}, y^{k+1}) \leq f(x^{k}, y^{k})
\end{equation}
so we have that the points of the sequence $\{(x^{k}, y^{k})\}$ belongs to the level set $\mathcal{L}_0$. Let $(x^*,y^*)$ be a limit point of $\{(x^k, y^k)\}$, i.e. there exist an infinite subset $K \subseteq N$ such that
\begin{equation}\label{eq:asim}
\lim_{k \in K, k \rightarrow \infty} (x^k, y^k) = (x^*,y^*) \qquad \lim_{k \in K, k \rightarrow \infty} d^{i(k),j(k)} = d^*
\end{equation}
By contradiction, let us assume that $(x^*,y^*)$ is not a solution. In this case, at least one of the following conditions hold:
\begin{subequations}
\begin{align}
&\nabla_y f(x^*,y^*) \neq 0  \label{eq:a}\\
&\exists \enskip i, j \enskip  \text{s.t.} \enskip x^*_i > l_i \enskip  \text{and} \enskip  \nabla_x f(x^*,y^*)^T d^{i,j} < 0 \label{eq:b}
\end{align}
\end{subequations}
For Point (\ref{eq:a}), we recall that for (\ref{eq:updatey}) $y^{k+1}$ minimizes $f(x^{k},y)$ with respect to $y$. So, if we imagine to perform Armijo along a descent direction (for example, $-\nabla_y f(x^{k},y^{k})$) we have
\begin{equation}\label{eq:dis}
f(x^{k}, y^{k+1}) \leq f(x^{k}, y^{k}) - \gamma \alpha \parallel \nabla_y f(x^{k}, y^{k}) \parallel ^2 \qquad \forall \alpha, \gamma > 0
\end{equation}
From (\ref{eq:dis}) and thanks to (\ref{eq:dec}), we can extend the Armijo convergence properties to $(x^{k+1}, y^{k+1})$. In particular, for Armijo we can write
\begin{equation}
\lim_{k \in K, k \rightarrow \infty} \parallel \nabla_y f(x^{k}, y^{k}) \parallel =  \parallel \nabla_y f(x^{*}, y^{*}) \parallel = 0
\end{equation}
So we have proved that (\ref{eq:a}) cannot hold.\\
For Point (\ref{eq:b}), the algorithm performs an Armijo line search along $d^{i(k),j(k)}$ to determine the step $\alpha^{k}$. Thanks to Armijo we can write
\begin{equation}\label{eq:armijoprop}
\lim_{k \in K, k \rightarrow \infty} \alpha^{k} \parallel d^{i(k),j(k)} \parallel \frac{ \left| \nabla_x f(x^{k}, y^{k+1})^T d^{i(k),j(k)} \right|}{\parallel d^{i(k),j(k)} \parallel} = 0
\end{equation}
From (\ref{eq:direction}), we know that $\parallel d^{i(k),j(k)} \parallel = \sqrt{2} \enskip\forall k$, so from (\ref{eq:armijoprop}) it follows
\begin{equation}\label{eq:armijoprop2}
\lim_{k \in K, k \rightarrow \infty} \alpha^{k}  \left| \nabla_x f(x^{k}, y^{k+1})^T d^{i(k),j(k)} \right| = 0
\end{equation}
By contradiction, assume that 
\begin{equation}\label{eq:wrong}
\lim_{k \in K, k \rightarrow \infty} \nabla_x f(x^k, y^k)^T d^{i(k),j(k)} = \nabla_x f(x^*, y^*)^T d^* = -\eta < 0
\end{equation}
(in particular, (\ref{eq:b}) implies (\ref{eq:wrong})).
From (\ref{eq:wrong}) and (\ref{eq:armijoprop2}) it follows that
\begin{equation}\label{eq:alpha}
\lim_{k \in K, k \rightarrow \infty} \alpha^k = 0
\end{equation}
In general, we have that
\begin{equation}
\lim_{k \in K, k \rightarrow \infty} \alpha^k \leq \lim_{k \in K, k \rightarrow \infty} \Delta^k
\end{equation}
We have to differentiate between two cases:
\begin{subequations}
\begin{align}
& \lim_{k \in K, k \rightarrow \infty} \Delta^k > 0 \label{eq:great}\\
& \lim_{k \in K, k \rightarrow \infty} \Delta^k = 0 \label{eq:zero}
\end{align}
\end{subequations}
If (\ref{eq:great}) holds, for large enough values of $k$, i.e. for $k \geq \hat{k}$, it must be $\alpha^k < \Delta^k$. From the Armijo properties we have, for $k \geq \hat{k}$:
\begin{equation}\label{eq:arm1}
f(x^k + \frac{\alpha^k}{\delta} d^{i(k),j(k)}, y^k) - f(x^k,y^k) > \gamma \frac{\alpha^k}{\delta} \nabla_x f(x^k, y^k)^T d^{i(k),j(k)}
\end{equation}
For the mean value theorem, we can write
\begin{equation}\label{eq:arm2}
f(x^k + \frac{\alpha^k}{\delta} d^{i(k),j(k)}, y^k) = f(x^k, y^k) + \frac{\alpha^k}{\delta} \nabla_x f(z^k, y^k)^T d^{i(k),j(k)}
\end{equation}
where $z^k = x^k + \vartheta_k \frac{\alpha^k}{\delta} d^{i(k),j(k)}$ and $\vartheta_k \in (0,1)$. From (\ref{eq:arm2}) and (\ref{eq:arm1}), for $k \geq \hat{k}$, we have
\begin{equation}\label{eq:nabla}
\nabla_x f(z^k, y^k)^T d^{i(k),j(k)}  > \gamma \nabla_x f(x^k , y^k)^T d^{i(k),j(k)}
\end{equation}
Using (\ref{eq:asim}) and (\ref{eq:alpha}) it must be
\begin{equation}
\lim_{k \in K, k \rightarrow \infty} z^k = \lim_{k \in K, k \rightarrow \infty} x^k + \vartheta_k \frac{\alpha^k}{\delta} d^{i(k),j(k)} = x^*
\end{equation}
Taking the limit value of each member of (\ref{eq:nabla}) we have
\begin{equation}
\nabla_x f(x^*, y^*)^T d^* \geq \gamma \nabla_x f(x^*, y^*)^T d^*
\end{equation}
From (\ref{eq:wrong}) we have
\begin{equation}
\eta \leq \gamma \eta
\end{equation}
that does not hold, since $\gamma < 1$. So we have proved that, if (\ref{eq:great}) holds, (\ref{eq:b}) cannot hold. \\
If (\ref{eq:zero}) holds, we can't assume that for $k \geq \hat{k}$ we have $\alpha^k < \Delta^k$. So we need to use Proposition (\ref{proposition:david}) that assures us that
there exists a number $M$ such that, for every $k \in K$, there exists an index $m(k)$ such that $k + m(k) \in K$, $0 \leq m(k) \leq M$ and 
\begin{equation}\label{eq:result}
\alpha^{k + m(k)} < \Delta^{k + m(k)}
\end{equation}
Thanks to (\ref{eq:result}), we can repeat the demonstration used in the case (\ref{eq:great}) and prove that, if (\ref{eq:zero}) holds, then (\ref{eq:b}) cannot hold.
\end{proof}