
\documentclass[preprint,12pt]{elsarticle}

\usepackage{amssymb}
\usepackage{amsmath,dsfont}
\usepackage{amsthm}
\usepackage[ruled,vlined, linesnumbered]{algorithm2e}
\usepackage{pgfplots}
\usepackage[utf8]{inputenc}
\usepackage{url}
\usepackage{tikz}
\usepackage{caption}
\usepackage{subfig}
\tikzset{
  font={\fontsize{9pt}{12}\selectfont}}
\newtheorem{theorem}{Theorem}[section]
\newtheorem{lemma}[theorem]{Lemma}
\newtheorem{oss}{Remark}

\newtheorem{proposition}[theorem]{Proposition}
\newtheorem{corollary}[theorem]{Corollary}
\newcommand{\R}{\mathbb{R}}
\newcommand{\N}{\mathbb{N}}
\DeclareMathOperator*{\argmin}{arg\,min}
\definecolor{gr}{HTML}{00BB00}
\setlength\parindent{0pt}

\journal{European Journal of Operational Research}

\begin{document}

\begin{frontmatter}

%% Title, authors and addresses

%% use the tnoteref command within \title for footnotes;
%% use the tnotetext command for theassociated footnote;
%% use the fnref command within \author or \address for footnotes;
%% use the fntext command for theassociated footnote;
%% use the corref command within \author for corresponding author footnotes;
%% use the cortext command for theassociated footnote;
%% use the ead command for the email address,
%% and the form \ead[url] for the home page:
%% \title{Title\tnoteref{label1}}
%% \tnotetext[label1]{}
%% \author{Name\corref{cor1}\fnref{label2}}
%% \ead{email address}
%% \ead[url]{home page}
%% \fntext[label2]{}
%% \cortext[cor1]{}
%% \address{Address\fnref{label3}}
%% \fntext[label3]{}

\title{Globally convergent decomposition algorithm for risk parity problem in portfolio selection}

%% use optional labels to link authors explicitly to addresses:
%% \author[label1,label2]{}
%% \address[label1]{}
%% \address[label2]{}

\author[mosek]{A.Cassioli\fnref{mail-cassioli}}
\author[firenze]{G.Cocchi\fnref{mail-cocchi}}
\author[firenze]{F.D'Amato\fnref{mail-damato}}
\author[firenze]{M.Sciandrone\fnref{mail-sciandrone}}

\address[mosek]{MOSEK ApS, Copenhagen, Denmark}
\address[firenze]{Dipartimento di Ingegneria dell'Informazione, Università di Firenze, Firenze, Italy}


\fntext[mail-cassioli]{andrea.cassioli@mosek.com}
\fntext[mail-cocchi]{guido.cocchi@unifi.it}
\fntext[mail-damato]{federico.damato@stud.unifi.it}
\fntext[mail-sciandrone]{marco.sciandrone@unifi.it}

\begin{abstract}
%% Text of abstract

\end{abstract}

\begin{keyword}
TODO\sep Portfolio\sep Risk-Parity \sep decomposition \sep optimization
%% keywords here, in the form: keyword \sep keyword

%% PACS codes here, in the form: \PACS code \sep code

%% MSC codes here, in the form: \MSC code \sep code
%% or \MSC[2008] code \sep code (2000 is the default)

\end{keyword}

\end{frontmatter}

%% \linenumbers

%% main text
\section{Introduction}
%%write something
\clearpage
\section{Preliminary background}\label{sect:2}
Let us consider the following optimization problem:
\begin{subequations}\label{eq:problem} 
\begin{align}
\min_{x,y} & \quad f(x,y)  \\
\text{s.t.} & \quad l \leq x \leq u \\
& \quad \mathds{1}^T x = 1 
\end{align}
\end{subequations}
where $x \in \R^n$, $y \in \R^m$, $f$ continuously differentiable, $l, u \in \R^n$ with $l < u$ and $\mathds{1} \in \R^n$ is all composed by ones. 

We define the feasible set $\mathcal{F}$  of Problem (\ref{eq:problem}):
\begin{equation}
\mathcal{F} = \{(x,y) \in \R^{n+m} : \mathds{1}^T x = 1, l \leq x \leq u\}.
\end{equation}
A vector $d\in \R^{n+m}$ is partitioned as follows
$$
d=\left(
\begin{array}{c}
d_x\\
d_y
\end{array}
\right ),
$$
where $d_x\in R^n$ and $d_y\in R^m$.

Given $(x,y) \in \mathcal{F}$, the set of feasible directions in $(x,y)$ is the cone
\begin{equation}
 \mathcal{D}(x,y)=\{ d \in \R^{n+m}: \mathds{1}^Td_x=0, d_i\ge 0 \ \forall i \in L(x), d_i\le 0 \ \forall i \in U(x)\}
\end{equation}
where
\begin{equation}
 \begin{aligned}
  &L(x)=\{ i: \ x_i=l_i\}\\
  &U(x)=\{ i: \ x_i=u_i\}
 \end{aligned}
\end{equation}
Given $(\bar x,\bar y) \in \mathcal{F}$, we say that $(\bar x,\bar y)$ is a {\it critical point} if
$$
\nabla f(\bar x,\bar y)^Td\ge 0\quad\quad \forall d\in  \mathcal{D}(\bar x,\bar y).
$$
We can state the following result.
\begin{proposition}\label{optimality}
A point $(\bar x,\bar y) \in \mathcal{F}$ is a critical point if and only if
\begin{equation}\label{on_x}
\begin{aligned}
&\nabla_xf(\bar x,\bar y)^Td_x\ge 0 \quad \forall d_x\in R^n & \\ 
&\text{s.t.} \quad \mathds{1}^Td_x=0,\quad d_i\ge 0 \ \forall i \in L(\bar x), \quad d_i\le 0 \ \forall i \in U(\bar x)&
\end{aligned}
\end{equation}
\begin{equation}\label{on_y}
 \nabla_yf(\bar x,\bar y)=0.
\end{equation} 
\end{proposition}

In correspondence to a feasible point $(x,y)$ we introduce the index sets
$$
R(x)=L(x)\cup C(x)
$$
$$
S(x)=U(x)\cup C(x),
$$
where 
$$
C(x)=\{i: l_i<x_i<u_i\}.
$$
We can state the following propositions (Propositions 2.2 and 2.3 in Jota2009).
\begin{proposition}\label{2.2}
 A feasible point $(\bar x,\bar y)$ is a critical point if and only if for each pair $(i,j)$,
$i\in R(\bar x)$, $j\in S(\bar x)$, we have
\begin{equation}\label{on_RS}
 {{\partial f(\bar x,\bar y)}\over{\partial x_i}}\ge
 {{\partial f(\bar x,\bar y)}\over{\partial x_j}}
\end{equation}
\begin{equation}\label{on_y2}
\nabla_yf(\bar x,\bar y)=0.
\end{equation}
\end{proposition}
\begin{proposition}\label{2.3}
 Let $\{(x^k,y^k)\}$ be a sequence of feasible points convergent to a point $(\bar x,\bar y)$.
Then, for sufficiently large values of $k$, we have
$$
R(\bar x)\subseteq R(x^k) \quad \quad {\rm and}\quad \quad S(\bar x)\subseteq S(x^k).
$$
\end{proposition}

\subsection{Set of sparse feasible directions}
In our decomposition framework we will employ feasible directions having only two
nonzero components.
Then, in this subsection we introduce these sparse feasible directions and we show their important properties.

Given $i, j\in  \{1, \ldots ,n\}$, with $i\ne j$,
we indicate by $d^{i,j}\in  \R^{n+m}$ such that
\begin{equation}\label{eq:direction}
d_h^{i,j}= 
\begin{cases}
1, \quad \text{    } h=i\\
-1, \text{    } \text{    } h=j\\
0, \quad \text{    } \text{otherwise}
\end{cases}
\end{equation}
Note that, by definition, the components $d_h^{i,j}$ with $h=n+1,\ldots,n+m$ are always set to zero.

Given $(x, y) \in \mathcal{F}$ and the corresponding index sets $R(x)$ and $S(x)$, we indicate by $D_{RS}(x,y)$
the set of directions $d^{i,j}$ with $i \in R(x)$ and $j \in S(x)$, namely
$$
D_{RS}(x,y)=\cup_{i\in R(x),j\in S(x)}d^{i,j}.
$$
\begin{proposition}\label{3.1}
Let $(\bar x,\bar y)$ be a feasible point. For each pair $i \in R(x)$ and $j \in S(x)$, the
direction $d^{i,j}\in \R^{n+m}$ is a feasible direction at $(\bar x,\bar y)$, i.e. $d \in D(\bar x,\bar y)$.
\end{proposition}
\begin{proposition}\label{3.2}
A feasible point $(\bar x,\bar y)$
 is a critical point if and only
\begin{equation}\label{on_x2}
\nabla f(\bar x,\bar y)^Td^{i,j}\ge 0\quad\quad \forall d^{i,j}\in D_{RS}(\bar x,\bar y)
\end{equation}
\begin{equation}\label{on_y3}
 \nabla_y f(\bar x,\bar y)=0.
\end{equation} 
\end{proposition}
Given a feasible point $(\bar x,\bar y)$, a pair $i\in R(\bar x)$ and $j\in S(\bar x)$ such that
$$
\nabla f(\bar x,\bar y)^Td^{i,j}<0
$$
is said a {\it Violating Pair} (VP).

A violating pair $(i^\star,j^\star)$ such that 
\begin{equation}\label{mvp}
 \nabla f(\bar x,\bar y)^Td^{i^\star,j^\star}\le \nabla f(\bar x,\bar y)^Td^{i,j} \quad \forall i\in R(\bar x), \ j\in S(\bar x).
\end{equation}
is the {\it Most Violating Pair} (MVP).



\clearpage
\section{A decomposition framework}
As already discussed, we partition the vector of variables into two blocks in order to take into account
the structure of the feasible set and possibly the form of the objective function (see, for instance,
the formulation of the risk parity problem, where the objective function is convex w.r.t. the scalar variable $\theta$).
The first block contains the constrained variables $x$, the second block contains
the unconstrained variables $y$. 

A first possibility can be that of defining a two-blocks Gauss-Seidel algorithm.
According to this scheme, at each iteration, the two component vectors $x$ and $y$ are
sequentially updated by performing  minimization steps (either exact or inexact) by  suitable descent techniques.
Globally convergent results of Gauss-Seidel algorithms (both exact and inexact) have been established in \cite{}, \cite{}, \cite{}.

We present here a block descent algorithm where a further level of decomposition is
introduced with respect to the block component $x$. 
More specifically, at each iteration, only two variables are updated, those corresponding
to a {\it Violating Pair}, by performing an inexact line search along a feasible and descent direction.
The adoption of a decomposition strategy with respect to the subvector $x$ is suitable
whenever the number $n$ of variables is large.

Note that the properties of the standard Armijo-type line search do not guarantee, without further assumptions
on the descent search direction $d^k$, that the distance between successive points tends to zero, which is a usual requirement of decomposition methods. This motivates the employment of the Quadratic Line Searck (QLS) defined in the appendix and based on the acceptance condition
$$
f((x^k,y^k)+\alpha^kd^k)\le f((x^k),y^k))-\gamma (\alpha^k)^2\|d^k\|^2.
$$
At every step $k$ we choose a random subset $W^k\subset \{1,..,n\}$ such as
\begin{equation}\label{eq:lambda}
\frac{|W^k|}{n} \times 100 = \lambda
\end{equation}
For each $w \in W^k$, we compute the partial derivative $\frac{\partial f(x,\theta)}{\partial x_w}$ and we select the MVP among the indexes in $W^k$. If we don't find a violating pair in $W^k$, we randomly add indexes until we find one, and we use this violating pair to build the descent direction. To assure the global convergence properties, we evaluate the MVP among the full gradient $\nabla_x f(x,\theta)$ every $M$ iterations. \\
The algorithm is formally described below.

\begin{algorithm}[ht]
 \KwData{Given the initial feasible point $(x^{0}, y^{0})$}
 Set $k = 0$\\
 \While{(not convergence)}{
 \eIf{$k \enskip \text{mod} \enskip M = 0$}
  {
  Let $(i(k), j(k))$ be the MVP\\ 
  }
  {
  Let $(i(k), j(k))$ be the MVP among indexes in $W^k$\\ 
  }
  Compute a step $\alpha^{k}$  along the direction $d^k=d^{i(k),j(k)}$ by QLS\\
  Set $x_{i(k)}^{k+1} = x_{i(k)}^{k} + \alpha^{k}$, $x_{j(k)}^{k+1} = x_{j(k)}^{k} - \alpha^{k}$  \\
  Compute $y^{k+1}$ such that $f(x^{k+1},y^{k+1})\le f(x^{k+1},y^k)$ and $\nabla_yf(x^{k+1},y^{k+1})=0$\\
  Set $k = k + 1$
 }
 \caption{Decomposition Algorithm}
\end{algorithm}
\subsection{Proximal Point approximation}
In this section we propose a version of previous algorithm which uses proximal point approximation.

Let us redefine selected variable $x_{i(k)},x_{j(k)}$ as $x_{i},x_{j}$ for ease of notation.

The idea behind proximal point modification is that at every iteration $k$ it's easy to solve the subproblem:
\begin{align}
 &\min_{x_i,x_j} f(x_i,x_j)+ \frac{1}{2} \tau||(x_i,x_j)-(x_i^k,x_j^k)||^2\\
 &x_i+x_j = \underbrace{1-\sum_{h \ne i,j} x^k_h}_{c^k}\\
 &l \le x_i,x_j\le u
 \end{align}
with $\tau>0$.

In fact thanks to simplex constraint, the objective function depends only on one of the selected variables (e.g. $x_j$). 
Additionally it is simply a 4-degree polynomial in $x_j$.
Because of a continuous function admits minimum in a compact set, we have to find it between zeros of derivative and the limit points of the feasible set.

Let us define $h'(\xi)$ as objective function derivative, then we can define set of possible global minima as:
\begin{equation}
 O_k = \{ \xi < x_j^k: h'(\xi)=0\} \cup \{\min\{l_j,x_i^k\} \}
\end{equation}

then we set:
\begin{equation}
x_j^{k+1}= \arg \min_{\xi \in O_k} \{h(\xi)\}
\end{equation}
and:
\begin{equation}
x_i^{k+1}= c^{k}-x^{k+1}_j
\end{equation}



\begin{algorithm}[ht]
 \KwData{Given the initial feasible point $(x^{0}, y^{0})$}
 Set $k = 0$\\
 \While{(not convergence)}{
 \eIf{$k \enskip \text{mod} \enskip M = 0$}
  {
  Let $(i(k), j(k))$ be the MVP\\ 
  }
  {
  Let $(i(k), j(k))$ be the MVP among indexes in $W^k$\\ 
  }
  Compute $\displaystyle x_{i(k)},x_{j(k)}\in \arg \min_{\xi,\zeta} f(\xi,\zeta)+\frac{1}{2}\tau ||(\xi,\zeta)-(x_{i(k)}^k,x_{j(k)}^k)||^2$\\
  Compute $y^{k+1}$ such that $f(x^{k+1},y^{k+1})\le f(x^{k+1},y^k)$ and $\nabla_yf(x^{k+1},y^{k+1})=0$\\
  Set $k = k + 1$
 }
 \caption{Decomposition Algorithm with proximal point}
\end{algorithm}


\clearpage
\section{Convergence analysis}
\begin{proposition}
Suppose that the level set $\mathcal{L}_0$ is a compact set. Let $\{(x^k, y^k)\}$ be the sequence of points generated by the decomposition algorithm. Then
\begin{itemize}
\item $(x^k, y^k) \in \mathcal{L}_0 \enskip \forall k$ 
\item  $\{(x^k, y^k)\}$ admits limit points and each limit point is a solution for Problem (\ref{eq:problem})
\end{itemize}
\end{proposition}
\begin{proof}
From (\ref{eq:updatey}) we have
\begin{equation}
f(x^{k}, y^{k+1}) \leq f(x^{k}, y^{k})
\end{equation}
and from the update rule of $x$ it follows (thanks to the Armijo properties)
\begin{equation}\label{eq:dec}
f(x^{k+1}, y^{k+1}) \leq f(x^{k}, y^{k+1})
\end{equation}
Finally we have
\begin{equation}
f(x^{k+1}, y^{k+1}) \leq f(x^{k}, y^{k})
\end{equation}
so we have that the points of the sequence $\{(x^{k}, y^{k})\}$ belongs to the level set $\mathcal{L}_0$. Let $(\overline{x},\overline{y})$ be a limit point of $\{(x^k, y^k)\}$, i.e. there exist an infinite subset $K \subseteq N$ such that
\begin{equation}\label{eq:asim}
\lim_{k \in K, k \rightarrow \infty} (x^k, y^k) = (\overline{x},\overline{y}) \qquad \lim_{k \in K, k \rightarrow \infty} d^{i(k),j(k)} = \overline{d}
\end{equation}
By contradiction, let us assume that $(\overline{x},\overline{y})$ is not a solution. In this case, at least one of the following conditions hold:
\begin{subequations}
\begin{align}
&\nabla_y f(\overline{x},\overline{y}) \neq 0  \label{eq:a}\\
&\exists \enskip i, j \enskip  \text{s.t.} \enskip \overline{x}_i > l_i \enskip  \text{and} \enskip  \nabla_x f(\overline{x},\overline{y})^T d^{i,j} < 0 \label{eq:b}
\end{align}
\end{subequations}
For Point (\ref{eq:a}), we recall that for (\ref{eq:updatey}) $y^{k+1}$ minimizes $f(x^{k},y)$ with respect to $y$. So, if we imagine to perform Armijo along a descent direction (for example, $-\nabla_y f(x^{k},y^{k})$) we have
\begin{equation}\label{eq:dis}
f(x^{k}, y^{k+1}) \leq f(x^{k}, y^{k}) - \gamma \alpha \parallel \nabla_y f(x^{k}, y^{k}) \parallel ^2 \qquad \forall \alpha, \gamma > 0
\end{equation}
From (\ref{eq:dis}) and thanks to (\ref{eq:dec}), we can extend the Armijo convergence properties to $(x^{k+1}, y^{k+1})$. In particular, for Armijo we can write
\begin{equation}
\lim_{k \in K, k \rightarrow \infty} \parallel \nabla_y f(x^{k}, y^{k}) \parallel =  \parallel \nabla_y f(\overline{x},\overline{y}) \parallel = 0
\end{equation}
So we have proved that (\ref{eq:a}) cannot hold.\\
For Point (\ref{eq:b}), the algorithm performs an Armijo line search along $d^{i(k),j(k)}$ to determine the step $\alpha^{k}$. Thanks to Armijo we can write
\begin{equation}\label{eq:armijoprop}
\lim_{k \in K, k \rightarrow \infty} \alpha^{k} \parallel d^{i(k),j(k)} \parallel \frac{ \left| \nabla_x f(x^{k}, y^{k+1})^T d^{i(k),j(k)} \right|}{\parallel d^{i(k),j(k)} \parallel} = 0
\end{equation}
From (\ref{eq:direction}), we know that $\parallel d^{i(k),j(k)} \parallel = \sqrt{2} \enskip\forall k$, so from (\ref{eq:armijoprop}) it follows
\begin{equation}\label{eq:armijoprop2}
\lim_{k \in K, k \rightarrow \infty} \alpha^{k}  \left| \nabla_x f(x^{k}, y^{k+1})^T d^{i(k),j(k)} \right| = 0
\end{equation}
By contradiction, assume that 
\begin{equation}\label{eq:wrong}
\lim_{k \in K, k \rightarrow \infty} \nabla_x f(x^k, y^k)^T d^{i(k),j(k)} = \nabla_x f(\overline{x},\overline{y})^T \overline{d} = -\eta < 0
\end{equation}
(in particular, (\ref{eq:b}) implies (\ref{eq:wrong})).
From (\ref{eq:wrong}) and (\ref{eq:armijoprop2}) it follows that
\begin{equation}\label{eq:alpha}
\lim_{k \in K, k \rightarrow \infty} \alpha^k = 0
\end{equation}
In general, we have that
\begin{equation}
\lim_{k \in K, k \rightarrow \infty} \alpha^k \leq \lim_{k \in K, k \rightarrow \infty} \Delta^k
\end{equation}
We have to differentiate between two cases:
\begin{subequations}
\begin{align}
& \lim_{k \in K, k \rightarrow \infty} \Delta^k > 0 \label{eq:great}\\
& \lim_{k \in K, k \rightarrow \infty} \Delta^k = 0 \label{eq:zero}
\end{align}
\end{subequations}
If (\ref{eq:great}) holds, for large enough values of $k$, i.e. for $k \geq \hat{k}$, it must be $\alpha^k < \Delta^k$. From the Armijo properties we have, for $k \geq \hat{k}$:
\begin{equation}\label{eq:arm1}
f(x^k + \frac{\alpha^k}{\delta} d^{i(k),j(k)}, y^k) - f(x^k,y^k) > \gamma \frac{\alpha^k}{\delta} \nabla_x f(x^k, y^k)^T d^{i(k),j(k)}
\end{equation}
For the mean value theorem, we can write
\begin{equation}\label{eq:arm2}
f(x^k + \frac{\alpha^k}{\delta} d^{i(k),j(k)}, y^k) = f(x^k, y^k) + \frac{\alpha^k}{\delta} \nabla_x f(z^k, y^k)^T d^{i(k),j(k)}
\end{equation}
where $z^k = x^k + \vartheta_k \frac{\alpha^k}{\delta} d^{i(k),j(k)}$ and $\vartheta_k \in (0,1)$. From (\ref{eq:arm2}) and (\ref{eq:arm1}), for $k \geq \hat{k}$, we have
\begin{equation}\label{eq:nabla}
\nabla_x f(z^k, y^k)^T d^{i(k),j(k)}  > \gamma \nabla_x f(x^k , y^k)^T d^{i(k),j(k)}
\end{equation}
Using (\ref{eq:asim}) and (\ref{eq:alpha}) it must be
\begin{equation}
\lim_{k \in K, k \rightarrow \infty} z^k = \lim_{k \in K, k \rightarrow \infty} x^k + \vartheta_k \frac{\alpha^k}{\delta} d^{i(k),j(k)} = \overline{x}
\end{equation}
Taking the limit value of each member of (\ref{eq:nabla}) we have
\begin{equation}
\nabla_x f(\overline{x},\overline{y})^T \overline{d} \geq \gamma \nabla_x f(\overline{x},\overline{y})^T \overline{d}
\end{equation}
From (\ref{eq:wrong}) we have
\begin{equation}
\eta \leq \gamma \eta
\end{equation}
that does not hold, since $\gamma < 1$. So we have proved that, if (\ref{eq:great}) holds, (\ref{eq:b}) cannot hold. \\
If (\ref{eq:zero}) holds, we can't assume that for $k \geq \hat{k}$ we have $\alpha^k < \Delta^k$. So we need to use Proposition (\ref{proposition:david}) that assures us that
there exists a number $M$ such that, for every $k \in K$, there exists an index $m(k)$ such that $k + m(k) \in K$, $0 \leq m(k) \leq M$ and 
\begin{equation}\label{eq:result}
\alpha^{k + m(k)} < \Delta^{k + m(k)}
\end{equation}
Thanks to (\ref{eq:result}), we can repeat the demonstration used in the case (\ref{eq:great}) and prove that, if (\ref{eq:zero}) holds, then (\ref{eq:b}) cannot hold.
\end{proof}

\clearpage
\section{Computational experiments}
%%write something
\clearpage
%% The Appendices part is started with the command \appendix;
%% appendix sections are then done as normal sections
%% \appendix

%% \section{}
%% \label{}

%% If you have bibdatabase file and want bibtex to generate the
%% bibitems, please use
%%
%%  \bibliographystyle{elsarticle-num} 
%%  \bibliography{<your bibdatabase>}

%% else use the following coding to input the bibitems directly in the
%% TeX file.

\begin{thebibliography}{00}
\section{Bibliography}
\bibitem{maillard}
S. Maillard, T. Roncalli, J. Teiletche,
\emph{On the properties of equally weighted risk contribution portfolios}, 
2010.

\bibitem{tutuncu}
  X. Bai, K. Scheinberg, R. Tutuncu,
  \emph{Least-square approach to risk parity in portfolio selection},
  2013.   
  
\bibitem{snopt}
	P. E. Gill, W. Murray, M. A. Sanders,
	\emph{SNOPT: An SQP Algorithm for large-scale constrained optimization}, 
	2005.


\end{thebibliography}
\end{document}
\endinput
%%
%% End of file `elsarticle-template-num.tex'.
