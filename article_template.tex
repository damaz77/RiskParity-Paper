
\documentclass[preprint,12pt]{elsarticle}

\usepackage{amssymb}
\usepackage{amsmath,dsfont}
\usepackage{amsthm}
\usepackage[ruled,vlined, linesnumbered]{algorithm2e}
\usepackage{pgfplots}
\usepackage[utf8]{inputenc}
\usepackage{url}
\usepackage{tikz}
\tikzset{
  font={\fontsize{9pt}{12}\selectfont}}
\newtheorem{theorem}{Theorem}[section]
\newtheorem{lemma}[theorem]{Lemma}
\newtheorem{proposition}[theorem]{Proposition}
\newtheorem{corollary}[theorem]{Corollary}
\newcommand{\R}{\mathbb{R}}
\newcommand{\N}{\mathbb{N}}
\DeclareMathOperator*{\argmin}{arg\,min}
\definecolor{gr}{HTML}{00BB00}
\setlength\parindent{0pt}

\journal{European Journal of Operational Research}

\begin{document}

\begin{frontmatter}

%% Title, authors and addresses

%% use the tnoteref command within \title for footnotes;
%% use the tnotetext command for theassociated footnote;
%% use the fnref command within \author or \address for footnotes;
%% use the fntext command for theassociated footnote;
%% use the corref command within \author for corresponding author footnotes;
%% use the cortext command for theassociated footnote;
%% use the ead command for the email address,
%% and the form \ead[url] for the home page:
%% \title{Title\tnoteref{label1}}
%% \tnotetext[label1]{}
%% \author{Name\corref{cor1}\fnref{label2}}
%% \ead{email address}
%% \ead[url]{home page}
%% \fntext[label2]{}
%% \cortext[cor1]{}
%% \address{Address\fnref{label3}}
%% \fntext[label3]{}

\title{Globally convergent decomposition algorithm for risk parity problem in portfolio selection}

%% use optional labels to link authors explicitly to addresses:
%% \author[label1,label2]{}
%% \address[label1]{}
%% \address[label2]{}

\author[mosek]{A.Cassioli\fnref{mail-cassioli}}
\author[firenze]{G.Cocchi\fnref{mail-cocchi}}
\author[firenze]{F.D'Amato\fnref{mail-damato}}
\author[firenze]{M.Sciandrone\fnref{mail-sciandrone}}

\address[mosek]{MOSEK ApS, Copenhagen, Denmark}
\address[firenze]{Dipartimento di Ingegneria dell'Informazione, Università di Firenze, Firenze, Italy}


\fntext[mail-cassioli]{andrea.cassioli@mosek.com}
\fntext[mail-cocchi]{guido.cocchi@unifi.it}
\fntext[mail-damato]{federico.damato@stud.unifi.it}
\fntext[mail-sciandrone]{marco.sciandrone@unifi.it}

\begin{abstract}
%% Text of abstract

\end{abstract}

\begin{keyword}
TODO\sep Portfolio\sep Risk-Parity \sep decomposition \sep optimization
%% keywords here, in the form: keyword \sep keyword

%% PACS codes here, in the form: \PACS code \sep code

%% MSC codes here, in the form: \MSC code \sep code
%% or \MSC[2008] code \sep code (2000 is the default)

\end{keyword}

\end{frontmatter}

%% \linenumbers

%% main text
\section{Introduction}
%%write something
\clearpage
\section{Preliminary background}\label{sect:2}
Let us consider the following optimization problem:
\begin{subequations}\label{eq:problem} 
\begin{align}
\min_{x,y} & \quad f(x,y)  \\
\text{s.t.} & \quad l \leq x \leq u \\
& \quad \mathbf{1}^T x = 1 
\end{align}
\end{subequations}
where $x \in \R^n$, $y \in \R^m$, $f$ continuously differentiable, $l, u \in \R^n$ with $l < u$ and $\mathbf{1} \in \R^n$ is all composed by ones. 

Now we define the feasible set $\mathcal{F}$  of Problem (\ref{eq:problem}):
\begin{equation}
\mathcal{F} = \{(x,y) \in \R^{n+m} : \mathbf{1}^T x = 1, l \leq x \leq u\}.
\end{equation}
Since the constraints of Problem (\ref{eq:problem}) respect constraints qualification conditions, a feasible point $(x,y)$ is a stationary point, if the Karush-Kuhn-Tucker (KKT) conditions are satisfied.

Let $L(x,y,\lambda,\mu,\gamma)$ the Lagrangian function associated to Problem (\ref{eq:problem}) then we can write KKT conditions.

\begin{proposition}[Optimality conditions (Necessary)]\label{prop:KKT}

Let $(x^*,y^*) \in \R^{n+m}$, with $(x^*,y^*) \in \mathcal{F}$, a local optimum for Problem (\ref{eq:problem}). Then there exist three multipliers $\lambda^* \in \R^n$, $\mu^* \in \R^n, \gamma^* \in \R$ such that:
\begin{equation}
 \begin{aligned}
  &\nabla_x L(x^*,y^*\lambda^*,\mu^*,\gamma^*)= \nabla_x F(x^*,y^*)+\lambda^*-\mu^*+\gamma^*=0\\
 &\nabla_x L(x^*,y^*\lambda^*,\mu^*,\gamma^*)=\nabla_y F(x^*,y^*) =0 \\
    &\lambda^*_i(l-x_i^*)=0,\ \forall i\\
 &\mu^*_i(x_i^*-u)=0,\ \forall i\\
   & \lambda^*,\mu^*\ge0 \\
 \end{aligned}
\end{equation}
\end{proposition}

From the first condition we have:
\begin{equation}
 \nabla_x f(x^*,y^*)-\lambda^*+\mu^*+\gamma^*=0
\end{equation}

Then there are three possible cases:
\begin{equation}
 \frac{\partial F(x^*,y^*)}{dx_i} = \begin{cases} -\mu_i^* -\gamma \hspace{1cm} x^*_i =u \\
 -\gamma+\lambda^*_i \hspace{1cm} x^*_i =l \\
 -\gamma \hspace{1.65cm} l<x^*_i <u 
\end{cases}
\end{equation}
Then if $x^*_i>l$: 
\begin{equation}
 \frac{\partial f(x^*,y^*)}{dx_i} \le \frac{\partial f(x^*,y^*)}{dx_h}, \forall h
\end{equation}

Let $(x,y) \in \mathcal{F}$, we define a set of all feasible direction in $(x,y)$:
\begin{equation}
 \mathcal{D}(x,y)=\{ d \in \R^{n+m}: \mathbf{1}^Td_x=0, d_i\ge 0 \ \forall i \in L(x), d_i\le 0 \ \forall i \in U(x)\}
\end{equation}
where:
\begin{equation}
 \begin{aligned}
  &L(x)=\{ i: \ x_i=l\}\\
  &U(x)=\{ i: \ x_i=u\}
 \end{aligned}
\end{equation}



\subsection{Set of sparse feasible directions}
Because of we will describe our decomposition method with respect to $x$ and $y$, we have to pay attention only on the $x$ variable to find a feasible descent direction w.r.t. $x$. 
In our case we want to build a set of sparse feasible direction in order to justify our decomposition approach.

Let $(x,y) \in \mathcal{F}$ non stationary w.r.t. $x$, then it's easy to see that
\begin{equation}
 L(x)\ne \{1,\ldots,n\}
\end{equation}
hence $\exists i,j$ such that $x_j>l_j$ and $i \ne j$ such that:
\begin{equation}
 \frac{\partial f(x)}{dx_j} > \frac{\partial f(x)}{dx_i}, 
\end{equation}

Now we define a direction $d^{i,j} \in \R^n$ with only two non-zero components such that:
\begin{equation}\label{eq:direction}
d_h^{i,j}= 
\begin{cases}
1, \quad \text{    } h=i\\
-1, \text{    } \text{    } h=j\\
0, \quad \text{    } \text{otherwise}
\end{cases}
\end{equation}

\begin{proposition}
Let $(x,y)$ a feasible point for Problem (\ref{eq:problem}). Then the direction $d^{i,j}$ is  feasible and descent direction in $x$.
\end{proposition}
\begin{proof}
For the feasibility it is enough to see that $\mathbf{1}^Td^{i,j}=1-1=0$.
Then we can apply sufficient conditions for descent direction in $x$, such that:
\begin{equation*}
 \nabla_xf(x,y)^Td^{i,j} =  \frac{\partial f(x)}{dx_i} - \frac{\partial f(x)}{dx_j}<0; 
\end{equation*}
\end{proof}

As always, we should choose the steepest descent direction composed by only two non-zero components.
This can be done computing the \emph{Most Violating Pair} $(i,j)$ such that $x_j>l_j$ and:
\begin{equation}
 (i,j) \in \arg \min_{l,m} \left\{\frac{\partial f(x)}{dx_l} - \frac{\partial f(x)}{dx_m}  \right\}
\end{equation}

If one doesn't want to use decomposition methods, he can define a direction $d_{xy} \in \R^{n+m}$ such that:
\begin{equation}
 d_{xy}=\{d^{i,j},-\nabla_yf(x,y)\}
\end{equation}

\subsection{Armijo-Type Line Search Algorithm}
In this section, we briefly describe the well-known Armijo-type line search along a feasible descent direction. The procedure will be used in the decomposition method presented in the next section. 
Let $d^{k} \in \mathcal{D}(x_k)$  $x^{k} \in \mathcal{F}$. In particular we choose $d^{k}=d^{i,j}_k$ with MVP $(i(k),j(k))$.
We denote by $\Delta_{k}$ the maximum feasible step along $d^{k}$. 

It is easy to see that:
\begin{equation*}
\Delta_k=\min \{ x^k_{j(k)}-l, u-x^k_{i(k)}\}
\end{equation*}
\begin{algorithm}[ht]
 \KwData{Given $\alpha > 0$, $\delta \in (0,1)$, $\gamma \in (0, 1/2)$ and the initial stepsize $\Delta^{(k)} =\min \{ x^k_{j(k)}-l, u-x^k_{i(k)}\}$ }
 %\KwResult{A feasible step $\lambda$}
 Set $\alpha = \Delta^{(k)}$\\
 \While{$f(x^{k},y^k) + \alpha d^{k}) > f(x^{k},y^k) + \gamma \alpha \nabla_x f(x^{k},y^k)^T d^{k}$}{
  Set $\alpha = \delta \alpha$
 }
 \caption{Armijo-Type Line Search}
\end{algorithm}
\iffalse
Then at iteration $k+1$ we have:
\begin{equation*}
x^{k+1}_{j(k)}=\begin{cases}
 l \ &\alpha_k=x^k_{j(k)}-l\\
 x^k_{j(k)}-u+x^k_{i(k)} \ &\alpha_k=u-x^k_{i(k)}
 \end{cases}
\end{equation*}
and:
\begin{equation*}
x^{k+1}_{i(k)}=\begin{cases}
 x^k_{j(k)}-l+x^k_{i(k)} \ &\alpha_k=x^k_{j(k)}-l\\
 u \ &\alpha_k=u-x^k_{i(k)}
 \end{cases}
\end{equation*}




\begin{proposition}\label{proposition:david}
 If we apply Armijo type line-search, using MVP and a descent direction $d_k^{i,j}$ then exists $N \in \N$ such that at most after $N$ consecutive iterations, $\alpha_k<\Delta_k$,i.e.:
 \begin{equation}
  a^{k+M}<\Delta^{k+M}
 \end{equation}
\end{proposition}
\begin{proof}
First of all we have to consider $l_i = l, u_i=u, \forall i$ and we define $l-norm (u-norm)$ of a vector, such that:
\begin{equation*}
\begin{aligned}
 ||x||_l:= |\{x_i| x_i >l\}| \\
 ||x||_u:= |\{x_i| x_i <u\}| 
 \end{aligned}
\end{equation*}
\end{proof}

The maximum feasible step at every iteration is
\begin{equation}
\Delta_k= \min \{ x^k_{j(k)}-l, u-x^k_{i(k)}\}
\end{equation}

Then, once renaming $i(k)=i,j(k)=j$, at iteration $k+1$ we have:
\begin{equation*}
x^{k+1}_{j}=\begin{cases}
 l \ &x^k_{i}\le u+l-x^k_{j}\\
 x^k_{j}-u+x^k_{i} \ &x^k_{i}> u+l-x^k_{j}
 \end{cases}
\end{equation*}
and:
\begin{equation*}
x^{k+1}_{i}=\begin{cases}
 x^k_{j}-l+x^k_{i} \ &x^k_{i}\le u+l-x^k_{j}\\
 u \ &x^k_{i}> u+l-x^k_{j}
 \end{cases}
\end{equation*}

Hence we have the follwing cases:
\begin{equation}
 ||x^{k+1}||_l=\begin{cases} ||x_{k}||_l\hspace{2cm} x^k_{i}> u+l-x^k_{j} \\
 ||x_k||_l-1\hspace{2cm} else               
              \end{cases}
\end{equation}
and:
\begin{equation}
 ||x^{k+1}||_u=\begin{cases} ||x_{k}||_u\hspace{2cm} x^k_{i}\le u+l-x^k_{j} \\
 ||x_k||_u-1\hspace{2cm} else               
              \end{cases}
\end{equation}

Because of the sequences of $||\cdot||_l, ||\cdot||_u$ are not increasing.
Indeed the two first cases hold for at most $2n!$ consecutive iteration (they are simple permutation of vector $x^k$ ) and the two second cases hold for at most $n-1$ consecutive iterations,
we can conclude that there is a number $M(k) \in \mathbb{N}$ such that:
\begin{equation}
 \alpha_{k+M(k)} < \Delta_{k+M(k)}
\end{equation}
and $M(k)\le 2(n-1)n!$.
\fi
\subsection{Quadratic Line Search}
Another possible line search method is quadratic line search, which is very useful in convergence analysis because it guarantees that the distance between two consecutive points converges to zero.
The QLS algorithm procedure is Armijo-like, but the step $\alpha_k$ is accepted if it satisfies:
\begin{equation}
f(x_k+\alpha_kd_k) \le  f(x_k)- \gamma (\alpha_k||d_k||)^2
\end{equation}
 where $d_k$ is a descent direction in $x_k$.

 \begin{algorithm}[ht]
 \KwData{Given $\alpha > 0$, $\delta \in (0,1)$, $\gamma \in (0, 1/2)$ and the initial stepsize $\Delta^{k} =\min \{ x^k_{j(k)}-l, u-x^k_{i(k)}\}$ }
 %\KwResult{A feasible step $\lambda$}
 Set $\alpha = \Delta^{k}$\\
 \While{$f(x^{k},y^k) + \alpha d^{k}) > f(x^{k},y^k) - \gamma \left(\alpha ||d^{k}||\right)^2$}{
  Set $\alpha = \delta \alpha$
 }
 \caption{QLS Line Search}
\end{algorithm}

\subsection{Exact Line Search}
When we move along the direction $d^{i(k),j(k)}$, defined in (\ref{eq:direction}), we modify only 2 variables ($x_{i(k)}, x_{j(k)}$) leaving the others unchanged. Thus, we can see our $f(x,y)$ as a function of two components, i.e. we can rewrite Problem (\ref{eq:problem}) as
\begin{subequations}\label{eq:twocomp} 
\begin{align}
\min_{x_{i(k)}, x_{j(k)}} & \quad f(x_{i(k)}, x_{j(k)})  \\
\text{s.t.} & \quad l_{i(k)} \leq x_{i(k)}  \leq u_{i(k)} \\
& \quad l_{j(k)} \leq x_{j(k)}  \leq u_{j(k)} \\
& \quad x_{i(k)}+x_{j(k)} = \underbrace{1-\sum_{h\ne {i(k)},{j(k)}}x_h}_c
\end{align}
\end{subequations}
Thanks to the last constraint, we can substitute $x_{i^*} = c - x_{j^*}$ and then we obtain
\begin{subequations}\label{eq:onecomp} 
\begin{align}
\min_{\xi} & \quad f(\xi) \\
\text{s.t.} & \quad x_{i(k)} = c - \xi \\
& \quad l_{\xi} \leq \xi \leq u_{\xi}
\end{align}
\end{subequations}
where:
\begin{equation}
\begin{aligned}
 l_{\xi} = \max\{l_{j(k)}, c - u_{i(k)}\}\\
 u_{\xi}= \min \{ u_{j(k)}, c-l_{i(k)}\}  
 \end{aligned}
\end{equation}

Because the domain is $I=[l_{\xi}, u_{\xi}]$, and $f(\xi)$ is continuous and differentiable in $I$, then $f$ has a minimum in $I$ and we can compute $f'(\xi)$. Let $R = \{ r \enskip | \enskip f'(r) = 0, r \in I \}$ be set set of the real feasible roots of $f'$. Each $r \in R$ can be a local maximum, minimum or flex; if $R = \{ \emptyset \}$, then the minimum of $f$ is on the extreme points of $I$.\\ 
Let $\xi^* = \argmin_{r \in R} f(r)$, then the optimal step $\alpha^*$ along the direction $d^{i(k),j(k)}$ is
\begin{equation}
\alpha_k^* = x_{j(k)} - \xi^* > 0
\end{equation} 

\clearpage
\section{A decomposition framework}
The algorithm follows the Gauss-Seidel scheme with 2 blocks of variables ($x$ and $y$). At each iteration, we optimize $f$ w.r.t. one block of variables, considering the other block fixed. \\
The $y$-block of variables is unconstrained, so we can find
\begin{equation}\label{eq:updatey}
y^{k+1} = \argmin_{y \in \mathbb{R}^m} f(x^{k},y)
\end{equation}
using the first-order necessary condition, i.e. \textit{TODO: ma questo vale se f è convessa rispetto ad y}
\begin{equation}
\nabla_y f(x^{k}, y^{k}) = 0
\end{equation}

\begin{algorithm}
 \KwData{Given the initial feasible guess $(x^{0}, y^{0})$}
 Set $k = 0$\\
 \While{(not convergence)}{
  Compute $y^{k+1}$ as in (\ref{eq:updatey})\\
  Compute $\nabla_{x} f(x^{k},y^{k+1})$ \\
  Choose indexes $i(k), j(k)$ using the Gauss-Southwell rules \\
  Choose a step $\alpha^{k}$  along the direction $d^{i(k),j(k)}$ \\
  Set $x^{k+1} = x^{k} + \alpha^{k}d^{i(k),j(k)}$ \\
  Set $k = k + 1$
 }
 \caption{Decomposition Algorithm}
\end{algorithm}

\clearpage
\section{Convergence analysis}
\begin{proposition}
Suppose that the level set $\mathcal{L}_0$ is a compact set. Let $\{(x^k, y^k)\}$ be the sequence of points generated by the decomposition algorithm. Then
\begin{itemize}
\item $(x^k, y^k) \in \mathcal{L}_0 \enskip \forall k$ 
\item  $\{(x^k, y^k)\}$ admits limit points and each limit point is a solution for Problem (\ref{eq:problem})
\end{itemize}
\end{proposition}
\begin{proof}
From (\ref{eq:updatey}) we have
\begin{equation}
f(x^{k}, y^{k+1}) \leq f(x^{k}, y^{k})
\end{equation}
and from the update rule of $x$ it follows (thanks to the Armijo properties)
\begin{equation}\label{eq:dec}
f(x^{k+1}, y^{k+1}) \leq f(x^{k}, y^{k+1})
\end{equation}
Finally we have
\begin{equation}
f(x^{k+1}, y^{k+1}) \leq f(x^{k}, y^{k})
\end{equation}
so we have that the points of the sequence $\{(x^{k}, y^{k})\}$ belongs to the level set $\mathcal{L}_0$. Let $(\overline{x},\overline{y})$ be a limit point of $\{(x^k, y^k)\}$, i.e. there exist an infinite subset $K \subseteq N$ such that
\begin{equation}\label{eq:asim}
\lim_{k \in K, k \rightarrow \infty} (x^k, y^k) = (\overline{x},\overline{y}) \qquad \lim_{k \in K, k \rightarrow \infty} d^{i(k),j(k)} = \overline{d}
\end{equation}
By contradiction, let us assume that $(\overline{x},\overline{y})$ is not a solution. In this case, at least one of the following conditions hold:
\begin{subequations}
\begin{align}
&\nabla_y f(\overline{x},\overline{y}) \neq 0  \label{eq:a}\\
&\exists \enskip i, j \enskip  \text{s.t.} \enskip \overline{x}_i > l_i \enskip  \text{and} \enskip  \nabla_x f(\overline{x},\overline{y})^T d^{i,j} < 0 \label{eq:b}
\end{align}
\end{subequations}
For Point (\ref{eq:a}), we recall that for (\ref{eq:updatey}) $y^{k+1}$ minimizes $f(x^{k},y)$ with respect to $y$. So, if we imagine to perform Armijo along a descent direction (for example, $-\nabla_y f(x^{k},y^{k})$) we have
\begin{equation}\label{eq:dis}
f(x^{k}, y^{k+1}) \leq f(x^{k}, y^{k}) - \gamma \alpha \parallel \nabla_y f(x^{k}, y^{k}) \parallel ^2 \qquad \forall \alpha, \gamma > 0
\end{equation}
From (\ref{eq:dis}) and thanks to (\ref{eq:dec}), we can extend the Armijo convergence properties to $(x^{k+1}, y^{k+1})$. In particular, for Armijo we can write
\begin{equation}
\lim_{k \in K, k \rightarrow \infty} \parallel \nabla_y f(x^{k}, y^{k}) \parallel =  \parallel \nabla_y f(\overline{x},\overline{y}) \parallel = 0
\end{equation}
So we have proved that (\ref{eq:a}) cannot hold.\\
For Point (\ref{eq:b}), the algorithm performs an Armijo line search along $d^{i(k),j(k)}$ to determine the step $\alpha^{k}$. Thanks to Armijo we can write
\begin{equation}\label{eq:armijoprop}
\lim_{k \in K, k \rightarrow \infty} \alpha^{k} \parallel d^{i(k),j(k)} \parallel \frac{ \left| \nabla_x f(x^{k}, y^{k+1})^T d^{i(k),j(k)} \right|}{\parallel d^{i(k),j(k)} \parallel} = 0
\end{equation}
From (\ref{eq:direction}), we know that $\parallel d^{i(k),j(k)} \parallel = \sqrt{2} \enskip\forall k$, so from (\ref{eq:armijoprop}) it follows
\begin{equation}\label{eq:armijoprop2}
\lim_{k \in K, k \rightarrow \infty} \alpha^{k}  \left| \nabla_x f(x^{k}, y^{k+1})^T d^{i(k),j(k)} \right| = 0
\end{equation}
By contradiction, assume that 
\begin{equation}\label{eq:wrong}
\lim_{k \in K, k \rightarrow \infty} \nabla_x f(x^k, y^k)^T d^{i(k),j(k)} = \nabla_x f(\overline{x},\overline{y})^T \overline{d} = -\eta < 0
\end{equation}
(in particular, (\ref{eq:b}) implies (\ref{eq:wrong})).
From (\ref{eq:wrong}) and (\ref{eq:armijoprop2}) it follows that
\begin{equation}\label{eq:alpha}
\lim_{k \in K, k \rightarrow \infty} \alpha^k = 0
\end{equation}
In general, we have that
\begin{equation}
\lim_{k \in K, k \rightarrow \infty} \alpha^k \leq \lim_{k \in K, k \rightarrow \infty} \Delta^k
\end{equation}
We have to differentiate between two cases:
\begin{subequations}
\begin{align}
& \lim_{k \in K, k \rightarrow \infty} \Delta^k > 0 \label{eq:great}\\
& \lim_{k \in K, k \rightarrow \infty} \Delta^k = 0 \label{eq:zero}
\end{align}
\end{subequations}
If (\ref{eq:great}) holds, for large enough values of $k$, i.e. for $k \geq \hat{k}$, it must be $\alpha^k < \Delta^k$. From the Armijo properties we have, for $k \geq \hat{k}$:
\begin{equation}\label{eq:arm1}
f(x^k + \frac{\alpha^k}{\delta} d^{i(k),j(k)}, y^k) - f(x^k,y^k) > \gamma \frac{\alpha^k}{\delta} \nabla_x f(x^k, y^k)^T d^{i(k),j(k)}
\end{equation}
For the mean value theorem, we can write
\begin{equation}\label{eq:arm2}
f(x^k + \frac{\alpha^k}{\delta} d^{i(k),j(k)}, y^k) = f(x^k, y^k) + \frac{\alpha^k}{\delta} \nabla_x f(z^k, y^k)^T d^{i(k),j(k)}
\end{equation}
where $z^k = x^k + \vartheta_k \frac{\alpha^k}{\delta} d^{i(k),j(k)}$ and $\vartheta_k \in (0,1)$. From (\ref{eq:arm2}) and (\ref{eq:arm1}), for $k \geq \hat{k}$, we have
\begin{equation}\label{eq:nabla}
\nabla_x f(z^k, y^k)^T d^{i(k),j(k)}  > \gamma \nabla_x f(x^k , y^k)^T d^{i(k),j(k)}
\end{equation}
Using (\ref{eq:asim}) and (\ref{eq:alpha}) it must be
\begin{equation}
\lim_{k \in K, k \rightarrow \infty} z^k = \lim_{k \in K, k \rightarrow \infty} x^k + \vartheta_k \frac{\alpha^k}{\delta} d^{i(k),j(k)} = \overline{x}
\end{equation}
Taking the limit value of each member of (\ref{eq:nabla}) we have
\begin{equation}
\nabla_x f(\overline{x},\overline{y})^T \overline{d} \geq \gamma \nabla_x f(\overline{x},\overline{y})^T \overline{d}
\end{equation}
From (\ref{eq:wrong}) we have
\begin{equation}
\eta \leq \gamma \eta
\end{equation}
that does not hold, since $\gamma < 1$. So we have proved that, if (\ref{eq:great}) holds, (\ref{eq:b}) cannot hold. \\
If (\ref{eq:zero}) holds, we can't assume that for $k \geq \hat{k}$ we have $\alpha^k < \Delta^k$. So we need to use Proposition (\ref{proposition:david}) that assures us that
there exists a number $M$ such that, for every $k \in K$, there exists an index $m(k)$ such that $k + m(k) \in K$, $0 \leq m(k) \leq M$ and 
\begin{equation}\label{eq:result}
\alpha^{k + m(k)} < \Delta^{k + m(k)}
\end{equation}
Thanks to (\ref{eq:result}), we can repeat the demonstration used in the case (\ref{eq:great}) and prove that, if (\ref{eq:zero}) holds, then (\ref{eq:b}) cannot hold.
\end{proof}

\clearpage
\section{Computational experiments}
%%write something
\clearpage
%% The Appendices part is started with the command \appendix;
%% appendix sections are then done as normal sections
%% \appendix

%% \section{}
%% \label{}

%% If you have bibdatabase file and want bibtex to generate the
%% bibitems, please use
%%
%%  \bibliographystyle{elsarticle-num} 
%%  \bibliography{<your bibdatabase>}

%% else use the following coding to input the bibitems directly in the
%% TeX file.

\begin{thebibliography}{00}
\section{Bibliography}
\bibitem{maillard}
S. Maillard, T. Roncalli, J. Teiletche,
\emph{On the properties of equally weighted risk contribution portfolios}, 
2010.

\bibitem{tutuncu}
  X. Bai, K. Scheinberg, R. Tutuncu,
  \emph{Least-square approach to risk parity in portfolio selection},
  2013.   
  
\bibitem{snopt}
	P. E. Gill, W. Murray, M. A. Sanders,
	\emph{SNOPT: An SQP Algorithm for large-scale constrained optimization}, 
	2005.


\end{thebibliography}
\end{document}
\endinput
%%
%% End of file `elsarticle-template-num.tex'.
