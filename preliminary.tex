We use volatility as risk measure of the fully invested portfolio, i.e.
\begin{equation}
\mathcal{R}(x) = \sigma(x) = \sqrt{x^T Q x}
\end{equation}
where $Q$ is the covariance matrix. Using the Euler decomposition, we can express the total risk as the sum of contributions from each asset in the portfolio:
\begin{equation}
\mathcal{R}(x) = \sum_{i=1}^n RC_i 
\end{equation}
where $RC_i$ is the risk contribution of the $i$-th asset, that has the form
\begin{equation}
RC_i = x_i \frac{\partial \mathcal{R}(x)}{\partial x_i}
\end{equation}
In the Risk Parity formulation, our aim is to satisfy the following set of constraints:
\begin{equation}\label{eq:rpconst}
x_i \frac{\partial \mathcal{R}(x)}{\partial x_i}= x_j \frac{\partial \mathcal{R}(x)}{\partial x_j} \quad \forall i,j
\end{equation}
We can also express (\ref{eq:rpconst}) in the following equivalent way:
\begin{equation}
x_i (Q x)_i = x_j (Q x)_j \quad \forall i,j
\end{equation}
In \cite{maillard} is proposed a least-square approach for solving the Risk Parity problem:
\begin{subequations}
\begin{align}
\min_x & \quad \sum_{i=i}^n \sum_{j=1}^{n}\left(x_i(Q x)_i - x_j(Q x)_j\right)^2\\
\text{s.t.} & \quad l \leq x \leq u \\
& \quad \mathds{1}^T x = 1 
\end{align}
\end{subequations}
The formulation proposed in \cite{tutuncu} introduces a free variable $y$ that is also optimized:
\begin{subequations}\label{eq:problem} 
\begin{align}
\min_{x,y} & \quad f(x,y) =  \sum_{i=i}^n \left(x_i(Q x)_i - y\right)^2 \\
\text{s.t.} & \quad l \leq x \leq u \\
& \quad \mathds{1}^T x = 1 
\end{align}
\end{subequations}
where $x \in \R^n$, $y \in \R$, $f$ continuously differentiable, $l, u \in \R^n$ with $l < u$ and $\mathds{1} \in \R^n$ is all composed by ones. 

Now we define the feasible set $\mathcal{F}$  of Problem (\ref{eq:problem}):
\begin{equation}
\mathcal{F} = \{(x,y) \in \R^{n+1} : \mathds{1}^T x = 1, l \leq x \leq u\}.
\end{equation}
Since the constraints of Problem (\ref{eq:problem}) respect constraints qualification conditions, a point $(x,y) \in \mathcal{F}$ is critical, if the Karush-Kuhn-Tucker (KKT) conditions are satisfied.

Let $L(x,y,\lambda,\mu,\gamma)$ the Lagrangian function associated to Problem (\ref{eq:problem}) then we can write KKT conditions.

\begin{proposition}[Optimality conditions (Necessary)]\label{prop:KKT}

Let $(x^*,y^*) \in \R^{n+1}$, with $(x^*,y^*) \in \mathcal{F}$, a local optimum for Problem (\ref{eq:problem}). Then there exist three multipliers $\lambda^* \in \R^n$, $\mu^* \in \R^n, \gamma^* \in \R$ such that:
\begin{equation}
 \begin{aligned}
  &\nabla_x L(x^*,y^*\lambda^*,\mu^*,\gamma^*)= \nabla_x f(x^*,y^*)+\lambda^*-\mu^*+\gamma^*=0\\
 &\nabla_y L(x^*,y^*,\lambda^*,\mu^*,\gamma^*)=\nabla_y f(x^*,y^*) =0 \\
    &\lambda^*_i(l_i-x_i^*)=0,\ \forall i\\
 &\mu^*_i(x_i^*-u_i)=0,\ \forall i\\
   & \lambda^*,\mu^*\ge0 \\
 \end{aligned}
\end{equation}
\end{proposition}

From the first condition we have:
\begin{equation}
 \nabla_x f(x^*,y^*)-\lambda^*+\mu^*+\gamma^*=0
\end{equation}

Then there are three possible cases:
\begin{equation}
 \frac{\partial f(x^*,y^*)}{dx_i} = \begin{cases} -\mu_i^* -\gamma^* \hspace{1cm} x^*_i =u \\
 -\gamma^*+\lambda^*_i \hspace{1cm} x^*_i =l \\
 -\gamma^* \hspace{1.65cm} l<x^*_i <u 
\end{cases}
\end{equation}
Then if $x^*_i>l_i$: 
\begin{equation}
 \frac{\partial f(x^*,y^*)}{dx_i} \le \frac{\partial f(x^*,y^*)}{dx_h}, \forall h
\end{equation}

After writing KKT optimality condition we focus on feasible direction in a feasibile point.

Let $(x,y) \in \mathcal{F}$, we define a set of all feasible direction in $(x,y)$:
\begin{equation}
 \mathcal{D}(x,y)=\{ d \in \R^{n+1}: \mathds{1}^Td_x=0, d_i\ge 0 \ \forall i \in L(x), d_i\le 0 \ \forall i \in U(x)\}
\end{equation}
where:
\begin{equation}
 \begin{aligned}
  &L(x)=\{ i: \ x_i=l_i\}\\
  &U(x)=\{ i: \ x_i=u_i\}
 \end{aligned}
\end{equation}

and $d_x \in \R^n$ represent direction $d$ with respect to $x$ variable.

\subsection{Set of sparse feasible directions}
Because of we will describe our decomposition method with respect to $x$ and $y$, we have to pay attention only on the $x$ variable in order to find a feasible descent direction in $(x,y)$.
Then when we will talk about line search algorithm, we always refer to $x$ variable.

In our case we want to build a set of sparse feasible direction in order to justify our decomposition approach.

Let $(x,y) \in \mathcal{F}$ non stationary w.r.t. $x$, then it's easy to see that
\begin{equation}
 L(x)\ne \{1,\ldots,n\}
\end{equation}
hence $\exists i,j$ such that $x_j>l_j$ and $i \ne j$ such that:
\begin{equation}
 \frac{\partial f(x)}{dx_j} > \frac{\partial f(x)}{dx_i}, 
\end{equation}

Now we define a direction $d^{i,j} \in \R^n$ with only two non-zero components such that:
\begin{equation}\label{eq:direction}
d_h^{i,j}= 
\begin{cases}
1, \quad \text{    } h=i\\
-1, \text{    } \text{    } h=j\\
0, \quad \text{    } \text{otherwise}
\end{cases}
\end{equation}

\begin{proposition}
Let $(x,y) \in \mathcal{F}$, then the direction $d^{i,j}$ is a feasible and descent direction in $x$.
\end{proposition}
\begin{proof}
For the feasibility it is enough to see that $\mathds{1}^Td^{i,j}=1-1=0$.
Then we can apply sufficient conditions for descent direction in $x$, such that:
\begin{equation*}
 \nabla_xf(x,y)^Td^{i,j} =  \frac{\partial f(x,y)}{dx_i} - \frac{\partial f(x,y)}{dx_j}<0; 
\end{equation*}
\end{proof}

As always, we should choose the steepest descent direction composed by only two non-zero components.
This can be done computing the \emph{Most Violating Pair} $(i,j)$ such that $x_j>l_j$ and:
\begin{equation}
 (i,j) \in \arg \min_{l,m} \left\{\frac{\partial f(x,y)}{dx_l} - \frac{\partial f(x,y)}{dx_m}  \right\}
\end{equation}

If we don't use decomposition methods, we can alternatively define a feasible descent direction $d \in \R^{n+1}$, in $(x,y)$ such that:
\begin{equation}
 d=\left(\begin{matrix}d^{i,j}\\
 -\nabla_yf(x,y)
   \end{matrix}\right)
\end{equation}
\iffalse
\subsection{Armijo-Type Line Search Algorithm}
In this section, we briefly describe the well-known Armijo-type line search along a feasible descent direction. The procedure will be used in the decomposition method presented in the next section. 
Let $d^{k} \in \mathcal{D}(x_k)$  $x^{k} \in \mathcal{F}$. In particular we choose $d^{k}=d^{i,j}_k$ with MVP $(i(k),j(k))$.
We denote by $\Delta_{k}$ the maximum feasible step along $d^{k}$. 

It is easy to see that:
\begin{equation*}
\Delta_k=\min \{ x^k_{j(k)}-l, u-x^k_{i(k)}\}
\end{equation*}
\begin{algorithm}[ht]
 \KwData{Given $\alpha > 0$, $\delta \in (0,1)$, $\gamma \in (0, 1/2)$ and the initial stepsize $\Delta^{(k)} =\min \{ x^k_{j(k)}-l, u-x^k_{i(k)}\}$ }
 %\KwResult{A feasible step $\lambda$}
 Set $\alpha = \Delta^{(k)}$\\
 \While{$f(x^{k},y^k) + \alpha d^{k}) > f(x^{k},y^k) + \gamma \alpha \nabla_x f(x^{k},y^k)^T d^{k}$}{
  Set $\alpha = \delta \alpha$
 }
 \caption{Armijo-Type Line Search}
\end{algorithm}

Then at iteration $k+1$ we have:
\begin{equation*}
x^{k+1}_{j(k)}=\begin{cases}
 l \ &\alpha_k=x^k_{j(k)}-l\\
 x^k_{j(k)}-u+x^k_{i(k)} \ &\alpha_k=u-x^k_{i(k)}
 \end{cases}
\end{equation*}
and:
\begin{equation*}
x^{k+1}_{i(k)}=\begin{cases}
 x^k_{j(k)}-l+x^k_{i(k)} \ &\alpha_k=x^k_{j(k)}-l\\
 u \ &\alpha_k=u-x^k_{i(k)}
 \end{cases}
\end{equation*}




\begin{proposition}\label{proposition:david}
 If we apply Armijo type line-search, using MVP and a descent direction $d_k^{i,j}$ then exists $N \in \N$ such that at most after $N$ consecutive iterations, $\alpha_k<\Delta_k$,i.e.:
 \begin{equation}
  a^{k+M}<\Delta^{k+M}
 \end{equation}
\end{proposition}
\begin{proof}
First of all we have to consider $l_i = l, u_i=u, \forall i$ and we define $l-norm (u-norm)$ of a vector, such that:
\begin{equation*}
\begin{aligned}
 ||x||_l:= |\{x_i| x_i >l\}| \\
 ||x||_u:= |\{x_i| x_i <u\}| 
 \end{aligned}
\end{equation*}
\end{proof}

The maximum feasible step at every iteration is
\begin{equation}
\Delta_k= \min \{ x^k_{j(k)}-l, u-x^k_{i(k)}\}
\end{equation}

Then, once renaming $i(k)=i,j(k)=j$, at iteration $k+1$ we have:
\begin{equation*}
x^{k+1}_{j}=\begin{cases}
 l \ &x^k_{i}\le u+l-x^k_{j}\\
 x^k_{j}-u+x^k_{i} \ &x^k_{i}> u+l-x^k_{j}
 \end{cases}
\end{equation*}
and:
\begin{equation*}
x^{k+1}_{i}=\begin{cases}
 x^k_{j}-l+x^k_{i} \ &x^k_{i}\le u+l-x^k_{j}\\
 u \ &x^k_{i}> u+l-x^k_{j}
 \end{cases}
\end{equation*}

Hence we have the follwing cases:
\begin{equation}
 ||x^{k+1}||_l=\begin{cases} ||x_{k}||_l\hspace{2cm} x^k_{i}> u+l-x^k_{j} \\
 ||x_k||_l-1\hspace{2cm} else               
              \end{cases}
\end{equation}
and:
\begin{equation}
 ||x^{k+1}||_u=\begin{cases} ||x_{k}||_u\hspace{2cm} x^k_{i}\le u+l-x^k_{j} \\
 ||x_k||_u-1\hspace{2cm} else               
              \end{cases}
\end{equation}

Because of the sequences of $||\cdot||_l, ||\cdot||_u$ are not increasing.
Indeed the two first cases hold for at most $2n!$ consecutive iteration (they are simple permutation of vector $x^k$ ) and the two second cases hold for at most $n-1$ consecutive iterations,
we can conclude that there is a number $M(k) \in \mathbb{N}$ such that:
\begin{equation}
 \alpha_{k+M(k)} < \Delta_{k+M(k)}
\end{equation}
and $M(k)\le 2(n-1)n!$.
\fi
\subsection{Quadratic Line Search (QLS)}
In order to find a feasible step along a descent direction in $x^k$, we use a line search method called quadratic line search.

It's easy to see that the maximum feasible step at every iteration $k$ is:
\begin{equation}
\Delta_k= \min \{ x^k_{j(k)}-l_{j(k)}, u_{i(k)}-x^k_{i(k)}\}
\end{equation}
The QLS algorithm procedure (algorithm \ref{alg:QLS}) starts from $\alpha_k = \Delta_k$ and decreases $\alpha_k$ until:
\begin{equation}
f(x_k+\alpha_kd_k) \le  f(x_k)- \gamma (\alpha_k||d_k||)^2
\end{equation}
 where $d_k$ is a descent direction in $x_k$.



Altough the most famous line search method is Armijo-Type, its most important drawback is that it can't guarantee that:
\begin{equation}
 \displaystyle \lim_{k\rightarrow \infty} ||x^{k+1}-x^{k}|| =0
\end{equation}



 
 \begin{algorithm}[ht]
 \KwData{Given $\alpha > 0$, $\delta \in (0,1)$, $\gamma \in (0, 1/2)$ and the initial stepsize $\Delta^{k} =\min \{ x^k_{j(k)}-l, u-x^k_{i(k)}\}$ }
 %\KwResult{A feasible step $\lambda$}
 Set $\alpha = \Delta^{k}$\\
 \While{$f(x^{k},y^k) + \alpha d^{k}) > f(x^{k},y^k) - \gamma \left(\alpha ||d^{k}||\right)^2$}{
  Set $\alpha = \delta \alpha$
 }
 \caption{QLS Line Search}\label{alg:QLS}
\end{algorithm}

QLS algorithm has also the same convergency properties \cite{sciandrone-galligari-dilorenzo} of Armijo-type.
\subsection{Exact Line Search (ELS)}
In our case when we move along the direction $d^{i(k),j(k)}$, defined in (\ref{eq:direction}), we modify only 2 variables ($x_{i(k)}, x_{j(k)}$) leaving the others unchanged. Thus, we can see our $f(x,y)$ as a function of two components, i.e. we can rewrite Problem (\ref{eq:problem}) as
\begin{subequations}\label{eq:twocomp} 
\begin{align}
\min_{x_{i(k)}, x_{j(k)}} & \quad f(x_{i(k)}, x_{j(k)})  \\
\text{s.t.} & \quad l_{i(k)} \leq x_{i(k)}  \leq u_{i(k)} \\
& \quad l_{j(k)} \leq x_{j(k)}  \leq u_{j(k)} \\
& \quad x_{i(k)}+x_{j(k)} = \underbrace{1-\sum_{h\ne {i(k)},{j(k)}}x_h}_c
\end{align}
\end{subequations}
Thanks to the last constraint, we can substitute $x_{i(k)} = c - x_{j(k)}$ and then we obtain
\begin{subequations}\label{eq:onecomp} 
\begin{align}
\min_{\xi} & \quad f(\xi) \\
\text{s.t.} & \quad x_{i(k)} = c - \xi \\
& \quad l_{\xi} \leq \xi \leq u_{\xi}
\end{align}
\end{subequations}
where:
\begin{equation}
\begin{aligned}
 l_{\xi} = \max\{l_{j(k)}, c - u_{i(k)}\}\\
 u_{\xi}= \min \{ u_{j(k)}, c-l_{i(k)}\}  
 \end{aligned}
\end{equation}

Because the domain is $I=[l_{\xi}, u_{\xi}]$, and $f(\xi)$ is continuous and differentiable in $I$, then $f$ has a minimum in $I$ and we can compute $f'(\xi)$. 

Let $R = \{ r \enskip | \enskip f'(r) = 0, r \in I \}$ be set set of the real feasible roots of $f'$. Each $r \in R$ can be a local maximum, minimum or flex; if $R = \{ \emptyset \}$, then the minimum of $f$ is on the extreme points of $I$.\\ 
Let $\xi^* =\displaystyle \argmin_{r \in R}\{ f(r)\}$, the optimal step $\alpha_k^*$ along the direction $d^{i(k),j(k)}$ is
\begin{equation}
\alpha_k^* = x_{j(k)} - \xi^* > 0
\end{equation} 
