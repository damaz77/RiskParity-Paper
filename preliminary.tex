Let us consider the following optimization problem:
\begin{subequations}\label{eq:problem} 
\begin{align}
\min_{x,y} & \quad f(x,y)  \\
\text{s.t.} & \quad l \leq x \leq u \\
& \quad \mathbf{1}^T x = 1 
\end{align}
\end{subequations}
where $x \in \R^n$, $y \in \R^m$, $f$ continuously differentiable, $l, u \in \R^n$ with $l < u$ and $\mathbf{1} \in \R^n$ is all composed by ones. 

Now we define the feasible set $\mathcal{F}$  of Problem (\ref{eq:problem}):
\begin{equation}
\mathcal{F} = \{(x,y) \in \R^{n+m} : \mathbf{1}^T x = 1, l \leq x \leq u\}.
\end{equation}
Since the constraints of Problem (\ref{eq:problem}) respect constraints qualification conditions, a feasible point $(x,y)$ is a stationary point, if the Karush-Kuhn-Tucker (KKT) conditions are satisfied.

Let $L(x,y,\lambda,\mu,\gamma)$ the Lagrangian function associated to Problem (\ref{eq:problem}) then we can write KKT conditions.

\begin{proposition}[Optimality conditions (Necessary)]\label{prop:KKT}

Let $(x^*,y^*) \in \R^{n+m}$, with $(x^*,y^*) \in \mathcal{F}$, a local optimum for Problem (\ref{eq:problem}). Then there exist three multipliers $\lambda^* \in \R^n$, $\mu^* \in \R^n, \gamma^* \in \R$ such that:
\begin{equation}
 \begin{aligned}
  &\nabla_x L(x^*,y^*\lambda^*,\mu^*,\gamma^*)= \nabla_x F(x^*,y^*)+\lambda^*-\mu^*+\gamma^*=0\\
 &\nabla_x L(x^*,y^*\lambda^*,\mu^*,\gamma^*)=\nabla_y F(x^*,y^*) =0 \\
    &\lambda^*_i(l-x_i^*)=0,\ \forall i\\
 &\mu^*_i(x_i^*-u)=0,\ \forall i\\
   & \lambda^*,\mu^*\ge0 \\
 \end{aligned}
\end{equation}
\end{proposition}

From the first condition we have:
\begin{equation}
 \nabla_x f(x^*,y^*)-\lambda^*+\mu^*+\gamma^*=0
\end{equation}

Then there are three possible cases:
\begin{equation}
 \frac{\partial F(x^*,y^*)}{dx_i} = \begin{cases} -\mu_i^* -\gamma \hspace{1cm} x^*_i =u \\
 -\gamma+\lambda^*_i \hspace{1cm} x^*_i =l \\
 -\gamma \hspace{1.65cm} l<x^*_i <u 
\end{cases}
\end{equation}
Then if $x^*_j>l$: \textit{TODO: io ho sempre fatto al contrario tra $i$ e $j$}
\begin{equation}
 \frac{\partial f(x^*)}{dx_j} \le \frac{\partial f(x^*)}{dx_i}, \forall i
\end{equation}

Let $(x,y) \in \mathcal{F}$, we define a set of all feasible direction in $(x,y)$:
\begin{equation}
 \mathcal{D}(x,y)=\{ d \in \R^{n+m}: \mathbf{1}^Td_x=0, d_i\ge 0 \ \forall i \in L(x), d_i\le 0 \ \forall i \in U(x)\}
\end{equation}
where:
\begin{equation}
 \begin{aligned}
  &L(x)=\{ i: \ x_i=l\}\\
  &U(x)=\{ i: \ x_i=u\}
 \end{aligned}
\end{equation}



\subsection{Set of sparse feasible directions}
Because of we will describe our decomposition method with respect to $x$ and $y$, we have to pay attention only on the $x$ variable to find a feasible descent direction w.r.t. $x$. 
In our case we want to build a set of sparse feasible direction in order to justify our decomposition approach.

Let $(x,y) \in \mathcal{F}$ non stationary w.r.t. $x$, then it's easy to see that
\begin{equation}
 L(x)\ne \{1,\ldots,n\}
\end{equation}
hence $\exists i$ such that $x_i>0$ and $j \ne i$ such that:
\begin{equation}
 \frac{\partial f(x)}{dx_i} > \frac{\partial f(x)}{dx_j}, 
\end{equation}

Now we define a direction $d^{i,j} \in \R^n$ with only two non-zero components such that:
\begin{equation}\label{eq:direction}
d_h^{i,j}= 
\begin{cases}
1, \quad \text{    } h=j\\
-1, \text{    } \text{    } h=i\\
0, \quad \text{    } \text{otherwise}
\end{cases}
\end{equation}

\begin{proposition}
Let $(x,y)$ a feasible point for Problem (\ref{eq:problem}). Then the direction $d^{i,j}$ is  feasible and descent direction in $x$.
\end{proposition}
\begin{proof}
For the feasibility it is enough to see that $\mathbf{1}^Td^{i,j}=1-1=0$.
Then we can apply sufficient conditions for descent direction in $x$, such that:
\begin{equation*}
 \nabla_xf(x,y)^Td^{i,j} =  \frac{\partial f(x)}{dx_j} - \frac{\partial f(x)}{dx_i}<0; 
\end{equation*}
\end{proof}

As always, we should choose the steepest descent direction composed by only two non-zero components.
This can be done computing the \emph{Most Violating Pair} $(i,j)$ such that $x_i>0$ and:
\begin{equation}
 (i,j) \in \arg \min_{l,m} \left\{\frac{\partial f(x)}{dx_l} - \frac{\partial f(x)}{dx_m}  \right\}
\end{equation}

If one doesn't want to use decomposition methods, he can define a direction $d_{xy} \in \R^{n+m}$ such that:
\begin{equation}
 d_{xy}=\{d^{i,j},-\nabla_yf(x,y)^
 \}
\end{equation}

\subsection{Armijo-Type Line Search Algorithm}
In this section, we briefly describe the well-known Armijo-type line search along a feasible descent direction. The procedure will be used in the decomposition method presented in the next section. 
Let $d^{k} \in \mathcal{D}(x_k)$  $x^{k} \in \mathcal{F}$. In particular we choose $d^{k}=d^{i,j}_k$ with MVP $(i(k),j(k))$.
We denote by $\Delta_{k}$ the maximum feasible step along $d^{k}$. 

It is easy to see that:
\begin{equation*}
\Delta_k=\min \{ x^k_{j(k)}-l, u-x^k_{i(k)}\}
\end{equation*}
Then at iteration $k+1$ we have:
\begin{equation*}
x^{k+1}_{j(k)}=\begin{cases}
 l \ &\alpha_k=x^k_{j(k)}-l\\
 x^k_{j(k)}-u+x^k_{i(k)} \ &\alpha_k=u-x^k_{i(k)}
 \end{cases}
\end{equation*}
and:
\begin{equation*}
x^{k+1}_{i(k)}=\begin{cases}
 x^k_{j(k)}-l+x^k_{i(k)} \ &\alpha_k=x^k_{j(k)}-l\\
 u \ &\alpha_k=u-x^k_{i(k)}
 \end{cases}
\end{equation*}


\begin{algorithm}[ht]
 \KwData{Given $\alpha > 0$, $\delta \in (0,1)$, $\gamma \in (0, 1/2)$ and the initial stepsize $\Delta^{(k)} =\min \{ x^k_{j(k)}-l, u-x^k_{i(k)}\}$ }
 %\KwResult{A feasible step $\lambda$}
 Set $\alpha = \Delta^{(k)}$\\
 \While{$f(x^{k},y^k) + \alpha d^{k}) > f(x^{k},y^k) + \gamma \alpha \nabla_x f(x^{k},y^k)^T d^{k}$}{
  Set $\alpha = \delta \alpha$
 }
 \caption{Armijo-Type Line Search}
\end{algorithm}

\begin{proposition}\label{proposition:david}
 If we apply Armijo type line-search, using MVP and a descent direction $d_k^{i,j}$ then exists $N \in \N$ such that at most after $N$ consecutive iterations, $\alpha_k<\Delta_k$,i.e.:
 \begin{equation}
  a^{k+M}<\Delta^{k+M}
 \end{equation}
\end{proposition}
\begin{proof}
First of all we have to consider $l_i = l, u_i=u, \forall i$ and we define $l-norm (u-norm)$ of a vector, such that:
\begin{equation*}
\begin{aligned}
 ||x||_l:= |\{x_i| x_i >l\}| \\
 ||x||_u:= |\{x_i| x_i <u\}| 
 \end{aligned}
\end{equation*}
\end{proof}

The maximum feasible step at every iteration is
\begin{equation}
\Delta_k= \min \{ x^k_{j(k)}-l, u-x^k_{i(k)}\}
\end{equation}

Then, once renaming $i(k)=i,j(k)=j$, at iteration $k+1$ we have:
\begin{equation*}
x^{k+1}_{j}=\begin{cases}
 l \ &x^k_{i}\le u+l-x^k_{j}\\
 x^k_{j}-u+x^k_{i} \ &x^k_{i}> u+l-x^k_{j}
 \end{cases}
\end{equation*}
and:
\begin{equation*}
x^{k+1}_{i}=\begin{cases}
 x^k_{j}-l+x^k_{i} \ &x^k_{i}\le u+l-x^k_{j}\\
 u \ &x^k_{i}> u+l-x^k_{j}
 \end{cases}
\end{equation*}

Hence we have the follwing cases:
\begin{equation}
 ||x^{k+1}||_l=\begin{cases} ||x_{k}||_l\hspace{2cm} x^k_{i}> u+l-x^k_{j} \\
 ||x_k||_l-1\hspace{2cm} else               
              \end{cases}
\end{equation}
and:
\begin{equation}
 ||x^{k+1}||_u=\begin{cases} ||x_{k}||_u\hspace{2cm} x^k_{i}\le u+l-x^k_{j} \\
 ||x_k||_u-1\hspace{2cm} else               
              \end{cases}
\end{equation}

Because of the sequences of $||\cdot||_l, ||\cdot||_u$ are not increasing.
Indeed the two first cases hold for at most $2n!$ consecutive iteration (they are simple permutation of vector $x^k$ ) and the two second cases hold for at most $n-1$ consecutive iterations,
we can conclude that there is a number $M(k) \in \mathbb{N}$ such that:
\begin{equation}
 \alpha_{k+M(k)} < \Delta_{k+M(k)}
\end{equation}
and $M(k)\le 2(n-1)n!$.

\subsection{Exact Line Search}
When we move along the direction $d^{i(k),j(k)}$, defined in (\ref{eq:direction}), we modify only 2 variables ($x_{i(k)}, x_{j(k)}$) leaving the others unchanged. Thus, we can see our $f(x,y)$ as a function of two components, i.e. we can rewrite Problem (\ref{eq:problem}) as
\begin{subequations}\label{eq:twocomp} 
\begin{align}
\min_{x_{i(k)}, x_{j(k)}} & \quad f(x_{i(k)}, x_{j(k)})  \\
\text{s.t.} & \quad l_{i(k)} \leq x_{i(k)}  \leq u_{i(k)} \\
& \quad l_{j(k)} \leq x_{j(k)}  \leq u_{j(k)} \\
& \quad x_{i(k)}+x_{j(k)} = \underbrace{1-\sum_{h\ne {i(k)},{j(k)}}x_h}_c
\end{align}
\end{subequations}
Thanks to the last constraint, we can substitute $x_{i^*} = c - x_{j^*}$ and then we obtain
\begin{subequations}\label{eq:onecomp} 
\begin{align}
\min_{\xi} & \quad f(\xi) \\
\text{s.t.} & \quad x_{i(k)} = c - \xi \\
& \quad ll_{j(k)} = \max\{l_{j(k)}, c - u_{i(k)}\} \leq \xi \leq \min \{ u_{j(k)}, c-l_{i(k)}\} = uu_{j(k)}
\end{align}
\end{subequations}
Because the domain is $I=[ll_{j(k)}, uu_{j(k)}]$, and $f(\xi)$ is continuous and differentiable in $I$, then $f$ has a minimum in $I$ and we can compute $f'(\xi)$. Let $R = \{ r \enskip | \enskip f'(r) = 0, r \in I \}$ be set set of the real feasible roots of $f'$. Each $r \in R$ can be a local maximum, minimum or flex; if $R = \{ \emptyset \}$, then the minimum of $f$ is on the extreme points of $I$.\\ 
Let $\xi^* = \argmin_{r \in R} f(r)$, then the optimal step $\alpha^*$ along the direction $d^{i(k),j(k)}$ is
\begin{equation}
\alpha_k^* = x_{j(k)} - \xi^* > 0
\end{equation} 
