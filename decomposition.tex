The algorithm follows the Gauss-Seidel scheme with 2 blocks of variables ($x$ and $y$). At each iteration, we optimize $f$ w.r.t. one block of variables, considering the other block fixed. \\
The $y$-block of variables is unconstrained, so we can find
\begin{equation}\label{eq:updatey}
y^{(k+1)} = \argmin_{y \in \mathbb{R}^m} f(x^{(k)},y)
\end{equation}
using the first-order necessary condition, i.e. \textit{TODO: ma questo vale se f è convessa rispetto ad y}
\begin{equation}
\nabla_y f(x^{(k)}, y^{(k)}) = 0
\end{equation}

\begin{algorithm}
 \KwData{Given the initial feasible guess $(x^{(0)}, y^{(0)})$}
 Set $k = 0$\\
 \While{(not convergence)}{
  Compute $y^{(k+1)}$ as in (\ref{eq:updatey})\\
  Compute $\nabla_{x} f(x^{(k)},y^{(k+1)})$ \\
  Choose indexes $i^*(k), j^*(k)$ using the Gauss-Southwell rules \\
  Choose a step $\alpha^{(k)}$  along the direction $d^{i^*(k),j^*(k)}$ \\
  Set $x^{(k+1)} = x^{(k)} + \alpha^{(k)}d^{i^*(k),j^*(k)}$ \\
  Set $k = k + 1$
 }
 \caption{Decomposition Algorithm}
\end{algorithm}
