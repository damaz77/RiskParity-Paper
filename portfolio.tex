In this section, we show that the general framework presented above can be used to a specific class of portfolio selection problem, namely the Risk Parity portfolio selection. Let $x \in \R^n$ be the portfolio, where $x_i$ represents the fraction of budget that is invested in the asset $i$, and let $Q$ be the covariance matrix between the assets; in \cite{bai} is introduced the following least-square optimization problem:
\begin{subequations}\label{eq:riskparity} 
\begin{align}
\min_{x,\theta} & \quad f(x,\theta) = \sum_{i=i}^n \left({x_i q_i^T \cdot x} - \theta \right)^2\\
\text{s.t.} & \quad l \leq x \leq u \\
& \quad \mathbf{1}^T x = 1 
\end{align}
\end{subequations}
where $\theta \in \R$ and $q_i$ is the $i$-th column of $Q$. Note that the problem is not necessary convex with respect to $x$ but it is strictly convex and coercive with respect to $\theta$. Infact we have:
\begin{equation}
\frac{\partial^2}{d^2\theta} f(x,\theta) = 2 
\end{equation}
which is a sufficient condition for a strictly convex function. At each iteration $k$, we have 
\begin{equation}
\theta^{k+1} = \argmin_\theta f(x^{k},\theta)
\end{equation}
$f$ is strictly convex with respect to $\theta$, so we can find $\theta^{k+1}$ as a solution of
\begin{equation}
\frac{\partial f(x^{k},\theta)}{d\theta} = 0 
\end{equation}
That is
\begin{equation}
\theta^{k+1} = \frac{\sum_{i=1}^n x_i^{k} q_i^T \cdot x^{k}}{n}
\end{equation}