In this section, we show that the general framework presented above can be used to a specific class of portfolio selection problem, namely the Risk Parity portfolio selection. We use volatility as risk measure of the fully invested portfolio, i.e.
\begin{equation}
\mathcal{R}(x) = \sigma(x) = \sqrt{x^T Q x}
\end{equation}
where $Q$ is the covariance matrix. Using the Euler decomposition, we can express the total risk as the sum of contributions from each asset in the portfolio:
\begin{equation}
\mathcal{R}(x) = \sum_{i=1}^n RC_i 
\end{equation}
where $RC_i$ is the risk contribution of the $i$-th asset, that has the form
\begin{equation}
RC_i = x_i \frac{\partial \mathcal{R}(x)}{\partial x_i}
\end{equation}
In the Risk Parity formulation, our aim is to satisfy the following set of constraints:
\begin{equation}\label{eq:rpconst}
x_i \frac{\partial \mathcal{R}(x)}{\partial x_i}= x_j \frac{\partial \mathcal{R}(x)}{\partial x_j} \quad \forall i,j
\end{equation}
We can also express (\ref{eq:rpconst}) in the following equivalent way:
\begin{equation}
x_i (Q x)_i = x_j (Q x)_j \quad \forall i,j
\end{equation}
In \cite{maillard} is proposed a least-square approach for solving the Risk Parity problem:
\begin{subequations}
\begin{align}
\min_x & \quad \sum_{i=i}^n \sum_{j=1}^{n}\left(x_i(Q x)_i - x_j(Q x)_j\right)^2\\
\text{s.t.} & \quad l \leq x \leq u \\
& \quad \mathds{1}^T x = 1 
\end{align}
\end{subequations}
The formulation proposed in \cite{tutuncu} introduces a free variable $y$ that is also optimized:
\begin{subequations}\label{eq:problemRP} 
\begin{align}
\min_{x,y} & \quad f(x,y) =  \sum_{i=i}^n \left(x_i(Q x)_i - y\right)^2 \\
\text{s.t.} & \quad l \leq x \leq u \\
& \quad \mathds{1}^T x = 1 
\end{align}
\end{subequations}
where $x \in \R^n$, $y \in \R$, $f$ continuously differentiable, $l, u \in \R^n$ with $l < u$ and $\mathds{1} \in \R^n$ is all composed by ones. \\
It is easy to see that Problem (\ref{eq:problemRP}) is a specific case of Problem (\ref{eq:problem}). Note that the formulation (\ref{eq:problemRP}) is non-convex.  If the optimization problem above has an optimal value of zero, then the RP portfolio is achieved. Otherwise, the value of $\sum_{i=i}^n \left( x_i(Q x)_i - y\right)^2$  can be regarded as a minimum variance measure towards our goal. Note that, since (\ref{eq:problemRP}) is a non-convex problem, in theory it is hard to solve and may produce local solutions. Anyway, we have the following lemma (for the proof see \cite{tutuncu}):
\begin{lemma}\label{rplemma}
Let $f(x,y) = \sum_{i=i}^n \left( x_i(Q x)_i - y \right)^2$. A solution pair $\{x,y\}$ is a global optimum with $f(x,y)=0$ if and only if $\nabla_xf(x,y) = 0$ and $\frac{\partial f(x,y)}{\partial y} = 0$.
\end{lemma}
Lemma (\ref{rplemma}) implies that if constraints of (\ref{eq:problemRP}) are not considered, then the first order optimality conditions determine the global optimal solution. On the other hand, when constraints are imposed, local optima and local stationary points can occur.